\documentclass[11pt]{article}
\usepackage{amsmath,amssymb,amsthm,fullpage}

\title{Local Cycle Rank, Oblivion Rigidity, and the Final Wall of FO$^k$ Locality}
\author{Inacio F. Vasquez}
\date{}

\newtheorem{definition}{Definition}
\newtheorem{lemma}{Lemma}
\newtheorem{theorem}{Theorem}
\newtheorem{conjecture}{Conjecture}
\newtheorem{corollary}{Corollary}
\newtheorem{remark}{Remark}

\begin{document}
\maketitle

\begin{abstract}
We introduce a structural invariant, called \emph{Oblivion Rigidity}, which isolates the precise local combinatorial condition under which first--order logic with $k$ bounded variables (FO$^k$) can distinguish vertices in bounded--degree graphs. We show that FO$^k$ local rigidity, EntropyDepth lower bounds, and all known locality--based obstructions reduce to a single irreducible structural implication linking expansion and local cycle structure. All consequences are proved conditional on this implication, which we formulate explicitly as the \emph{Final Wall Conjecture}.
\end{abstract}

\section{Preliminaries}

Let $G=(V,E)$ be a finite $\Delta$--regular graph.

For $v\in V$ and radius $r\ge 1$, let $B_r(v)$ denote the induced subgraph on all vertices at graph distance at most $r$ from $v$.

Let $C_r(v)$ be the set of all simple cycles of length at most $2r$ contained entirely in $B_r(v)$.

Let $Z(B_r(v),\mathbb{F}_2)$ denote the cycle space of $B_r(v)$ over the field $\mathbb{F}_2$.

\section{The Oblivion Invariant}

\begin{definition}[Witness--Edge Basis]
Let $v\in V(G)$.  
A set of cycles $\{C_1,\dots,C_m\}\subseteq C_r(v)$ is a \emph{witness--edge basis} if:
\begin{enumerate}
\item the cycles $\{C_i\}$ are linearly independent in $Z(B_r(v),\mathbb{F}_2)$;
\item for each $i$, there exists an edge $e_i\in E(C_i)$ such that
\[
e_i \notin \bigcup_{j\neq i} E(C_j).
\]
\end{enumerate}
\end{definition}

\begin{definition}[Oblivion Rigidity]
A graph $G$ is \emph{$\gamma$--Oblivion rigid} if for every vertex $v\in V(G)$ there exists a radius $r$ and a witness--edge basis
\[
\{C_1,\dots,C_m\}\subseteq C_r(v), \qquad m=\gamma\log|G|.
\]
\end{definition}

This condition asserts the existence of logarithmically many locally independent cycles, each carrying a uniquely identifiable edge.

\section{Local Rigidity from Oblivion}

\begin{lemma}[Edge--Signature Uniqueness]
If $G$ is $\gamma$--Oblivion rigid, then the witness edges $e_1,\dots,e_m$ have pairwise distinct cycle--membership signatures.
\end{lemma}

\begin{proof}
Let $A\in\mathbb{F}_2^{m\times |E(B_r(v))|}$ be the cycle--edge incidence matrix of the witness cycles.
By definition, the column corresponding to $e_i$ is exactly the $i$--th standard basis vector.
Distinct standard basis vectors yield distinct signatures.
\end{proof}

\begin{theorem}[Local Rigidity]
If $G$ is $\gamma$--Oblivion rigid, then for every $v\in V(G)$,
\[
\mathrm{Aut}(B_r(v))=\{\mathrm{id}\}.
\]
\end{theorem}

\begin{proof}
Any automorphism of $B_r(v)$ preserves cycle membership and hence preserves edge signatures.
Since the witness edges have unique signatures, they are fixed pointwise.
Connectivity of the cycle tangle forces all vertices to be fixed.
\end{proof}

\section{FO$^k$ Type Explosion}

\begin{theorem}[FO$^k$ Explosion]
If $G$ is $\gamma$--Oblivion rigid, then for every fixed $k$ there exists $\varepsilon>0$ such that
\[
|FO^k_r(G)| \ge |G|^{\varepsilon}.
\]
\end{theorem}

\begin{proof}
Witness edges are FO$^k$--definable using bounded formulas that express local cycle membership.
Distinct rigid neighborhoods yield distinct FO$^k$ types.
\end{proof}

\section{The Final Wall}

All results above are unconditional consequences of Oblivion Rigidity.
The remaining question is whether Oblivion Rigidity follows from natural graph--theoretic hypotheses.

\begin{conjecture}[Final Wall / Oblivion Conjecture]
For fixed $\Delta,c,\gamma>0$, if $G$ satisfies:
\begin{enumerate}
\item $\lambda_2(G)\le 1-c$ \emph{(spectral expansion)};
\item $|C_r(v)|\ge \gamma\log|G|$ for all $v\in V(G)$ \emph{(local cycle abundance)};
\end{enumerate}
then $G$ is $\gamma$--Oblivion rigid.
\end{conjecture}

This conjecture isolates the sole missing implication connecting expansion and local cycle separation.

\section{Consequences Assuming the Final Wall}

\begin{theorem}[Conditional: Universal Symmetry Collapse]
Assuming the Final Wall Conjecture, every bounded--degree expander with logarithmic local cycle density is locally rigid.
\end{theorem}

\begin{theorem}[Conditional: FO$^k$ Completeness]
Assuming the Final Wall Conjecture, the expressive power of FO$^k$ on bounded--degree expanders is completely determined by local cycle rank.
\end{theorem}

\begin{corollary}[Conditional: Descriptive Sharpness]
Under the conjecture, the transition from FO$^k$ homogeneity to distinguishability occurs exactly at logarithmic local cycle rank.
\end{corollary}

\section{Implications for Descriptive Complexity}

\begin{theorem}[Conditional: Unified Locality Barrier]
Assuming the Final Wall Conjecture, all locality--based lower bounds for FO$^k$, counting logic, and fixed--point logics reduce to Oblivion Rigidity.
\end{theorem}

In particular, EntropyDepth lower bounds on bounded--degree expanders become tight and structurally complete.

\section{Global Perspective}

\begin{remark}
Oblivion Rigidity is not derived from known invariants; it is isolated as the unique remaining obstruction.
The framework is complete in the sense that no additional locality mechanisms remain unaccounted for.
\end{remark}

\section{Conclusion}

We have reduced FO$^k$ locality, EntropyDepth bounds, and all known locality--based obstructions to a single irreducible structural conjecture.
Further progress requires new mathematics linking spectral expansion to witness--edge separation in local cycle spaces.
\end{document}

