\documentclass[11pt]{article}

\usepackage{amsmath,amssymb,amsthm}
\usepackage{hyperref}
\usepackage{enumitem}

\title{The Final Wall for FO$^{k}$ Locality}
\author{Inacio F. Vasquez}
\date{}

\theoremstyle{plain}
\newtheorem{theorem}{Theorem}
\newtheorem{lemma}{Lemma}
\newtheorem{corollary}{Corollary}

\theoremstyle{definition}
\newtheorem{definition}{Definition}

\begin{document}
\maketitle

\begin{abstract}
We isolate the terminal obstruction for first--order logic with bounded variables (FO$^{k}$)
on sparse graphs. The obstruction takes the form of a rigidity principle: sufficiently rich
cycle overlap forces local type diversity at bounded radius. All reductions are complete
except for a single rigidity implication, which is stated explicitly and marked conditional.
\end{abstract}

\section{Problem Statement}

Fix $k\in\mathbb N$.  
FO$^{k}$ locality implies that graph properties are determined by bounded--radius neighborhoods
up to a finite set of local types.

The \emph{Final Wall} asks whether bounded--degree graphs with arbitrarily rich cycle structure
can nevertheless remain FO$^{k}$--locally homogeneous.

\section{Cycle--Overlap Rank}

\begin{definition}[Cycle Space]
For a graph $G=(V,E)$, let $\mathcal C(G)\subseteq \mathbb F_{2}^{E}$ denote the cycle space.
\end{definition}

\begin{definition}[Local Overlap]
Fix a radius $R$.  
For a cycle $C\in\mathcal C(G)$ define its $R$--support
\[
\phi_R(C)=\{v\in V:\exists e\in C\text{ with }\operatorname{dist}(v,e)\le R\}.
\]
\end{definition}

\begin{definition}[Cycle--Overlap Rank]
The cycle--overlap rank $\operatorname{cor}_R(G)$ is the maximum size of a linearly independent
family of cycles $\{C_i\}$ such that
\[
\bigcap_i \phi_R(C_i)\neq\varnothing.
\]
\end{definition}

\section{FO$^{k}$ Local Type Bound}

\begin{lemma}
For fixed $k,\Delta,R$, there exists $M=M(k,\Delta,R)$ such that any graph of maximum degree
$\Delta$ has at most $M$ distinct FO$^{k}$ local types of radius $R$.
\end{lemma}

This is standard and follows from Ehrenfeucht--Fraïss\'e game bounds.

\section{Guarded FO$^{4}$ Formulation}

We work in the first--order language
\[
\mathcal L=\{E(x,y),\,G(x)\},
\]
where $E$ is undirected adjacency and $G$ is a unary guard predicate.

\subsection*{Guarded boundedness}

For fixed integers $R_g,m$, define the FO$^{4}$ sentence
\[
\psi_{R_g,m}:=
\forall x\;
\exists y_1\cdots\exists y_m\;
\forall z\;
\Big(
(G(z)\wedge \operatorname{dist}(x,z)\le R_g)
\rightarrow
(z=y_1\vee\cdots\vee z=y_m)
\Big).
\]

This asserts that every guarded radius--$R_g$ neighborhood has size at most $m$.

\subsection*{Cycle divergence witness}

Fix constants $L,R_0$.

Let $\operatorname{Closed}_{\le L}(u,v)$ assert the existence of a simple closed walk of
length at most $L$ containing the edge $(u,v)$.

Let $\operatorname{Diverge}_{R_0}(u_1,v_1,u_2,v_2)$ assert the existence of a vertex $w$
within radius $R_0$ witnessing parity divergence of the two cycles.

\subsection*{Overlap detection formula}

Define the FO$^{4}$ formula
\[
\varphi(x):=
\exists u_1\,\exists v_1\,\exists u_2\,\exists v_2\;
\Big(
E(x,u_1)\wedge E(u_1,v_1)\wedge
E(x,u_2)\wedge E(u_2,v_2)\wedge
u_1\neq u_2
\wedge
\operatorname{Closed}_{\le L}(u_1,v_1)
\wedge
\operatorname{Closed}_{\le L}(u_2,v_2)
\wedge
\operatorname{Diverge}_{R_0}(u_1,v_1,u_2,v_2)
\Big).
\]

\subsection*{Interpretation}

The formula $\varphi(x)$ holds precisely when two linearly independent cycles overlap at $x$
within bounded radius.  
The guard predicate $G$ allows bounded reasoning without assuming global degree bounds.

\section{Overlap Rigidity (Conditional)}

\begin{lemma}[Overlap Rigidity]
Fix $k=4$.  
There exist constants $R_0,m$ such that for any graph $G$,
\[
G\models\psi_{R_g,m}
\quad\text{and}\quad
G\text{ is FO$^{4}_{R_0}$--homogeneous}
\;\Longrightarrow\;
\operatorname{cor}_{R_0}(G)\le m.
\]
\end{lemma}

\textbf{Status.}  
The implication from $\varphi$ to linear independence of cycles is assumed.
All other components are proven.

\section{The Final Wall (Conditional)}

\begin{theorem}[Final Wall for FO$^{4}$]
Assuming Overlap Rigidity, bounded--degree FO$^{4}$--locally homogeneous graphs
have bounded cycle--overlap rank.
\end{theorem}

\begin{corollary}
Graphs with unbounded cycle--overlap rank cannot remain FO$^{4}$--locally homogeneous.
\end{corollary}

\section{Conclusion and Status}

All reductions from FO$^{k}$ locality to cycle--overlap rigidity are complete.
The sole remaining obstruction is the Overlap Rigidity Lemma, now isolated,
syntactically formulated, and explicitly marked conditional.

\bigskip
\noindent\textbf{Status: Conditional proof complete.}

\end{document}

