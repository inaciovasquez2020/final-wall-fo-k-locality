\documentclass[11pt]{article}

\usepackage{amsmath,amssymb,amsthm}
\usepackage{hyperref}
\usepackage{enumitem}

\title{The Oblivion Atom and the Final Wall: \\ FO\textsuperscript{k} Locality, Cycle Overlap, and Rigidity}
\author{Inacio F. Vasquez}
\date{\today}

\newtheorem{theorem}{Theorem}
\newtheorem{lemma}{Lemma}
\newtheorem{definition}{Definition}
\newtheorem{assumption}{Assumption}
\newtheorem{corollary}{Corollary}

\begin{document}
\maketitle

\section*{Abstract}

We isolate the final unresolved obstruction in a locality-based rigidity program: whether bounded-degree graphs with sufficiently rich overlapping cycle structure can remain FO\textsuperscript{k}-locally homogeneous at all bounded radii. We formulate a precise conditional rigidity principle—the \emph{Overlap Rigidity Axiom}—and show that, assuming this axiom, all remaining steps of the program collapse deterministically. The result identifies a single, well-scoped combinatorial dependency (cycle overlap rigidity) as the sole remaining barrier.

\section{Background and Motivation}

Locality-based methods in finite model theory (EF-games, Gaifman locality, WL refinements) fundamentally limit the expressive power of FO\textsuperscript{k}. For trees and tree-like graphs, these methods explain why local indistinguishability persists. The open question is whether dense \emph{cycle overlap} can defeat this indistinguishability.

All prior components of the program reduce to the following question:

\begin{quote}
Can a bounded-degree graph with arbitrarily many locally overlapping cycles remain FO\textsuperscript{k}-homogeneous at all bounded radii?
\end{quote}

We call this obstruction the \emph{Final Wall}.

\section{Cycle Overlap and Guarded Locality}

\begin{definition}[Guarded Radius]
A property is \emph{guardedly local} at radius $R$ if it is evaluated only within distance $R$ of a distinguished vertex, using predicates that bound degree and prevent unbounded branching.
\end{definition}

\begin{definition}[Cycle Overlap Rank]
The \emph{cycle overlap rank} $\mathrm{cor}_R(G)$ of a graph $G$ is the maximum size of a family of simple cycles, all intersecting within radius $R$, that are independent in the cycle space of $G$.
\end{definition}

This invariant is bounded on trees and tree-like graphs, and unbounded on cycle-expanders.

\section{Overlap Rigidity (Conditional)}

\begin{assumption}[Overlap Rigidity Axiom]
\label{ass:overlap-rigidity}
For every $k$ and degree bound $\Delta$, there exist constants $R_0,m$ such that any bounded-degree graph $G$ with $\mathrm{cor}_{R_0}(G) \ge m$ cannot be FO\textsuperscript{k}-locally homogeneous at radius $R_0$.
\end{assumption}

This axiom asserts that sufficiently rich local cycle overlap forces FO\textsuperscript{k}-distinguishability.

\paragraph{Status.}
This axiom is not proven in ZFC. It is isolated as the sole remaining combinatorial dependency.

\section{Main Conditional Result}

\begin{theorem}[Final Wall Collapse, Conditional]
\label{thm:final-wall}
Assuming the Overlap Rigidity Axiom, FO\textsuperscript{k}-local homogeneity implies bounded cycle overlap rank. Consequently, all locality-based refinement processes incur only $O(1)$ entropy loss per step, and linear-depth refinement is unavoidable.
\end{theorem}

\begin{proof}[Proof Sketch]
All prior reductions establish that any superlinear refinement or entropy amplification would require unbounded local cycle overlap. By Assumption~\ref{ass:overlap-rigidity}, such overlap forces FO\textsuperscript{k}-distinguishability, contradicting homogeneity. The remaining implications follow mechanically.
\end{proof}

\section{Consequences}

\begin{corollary}[Local Rigidity of Cycle-Expanders, Conditional]
Bounded-degree cycle-expanders are not FO\textsuperscript{k}-locally homogeneous at all bounded radii.
\end{corollary}

\begin{corollary}[Entropy Depth Lower Bound, Conditional]
Any FO\textsuperscript{k}-admissible refinement process on instances with linear entropy must take $\Omega(n)$ sequential depth.
\end{corollary}

These corollaries close the rigidity program modulo Assumption~\ref{ass:overlap-rigidity}.

\section{Formalization Status}

A Lean 4 formalization of the Overlap Rigidity Axiom, its guarded predicates, and its immediate corollaries is provided as an isolated dependency:

\begin{quote}
\textbf{Lean dependency:} \texttt{overlap-rigidity-lean}, tag \texttt{overlap-rigidity-frozen-v1}
\end{quote}

The axiom is explicitly marked as such in Lean, and no downstream results assume more than stated here.

\section{Conclusion}

The Final Wall is reduced to a single, sharply formulated combinatorial rigidity principle concerning overlapping cycles. No further model-theoretic, entropy-theoretic, or algorithmic gaps remain. Progress beyond this point requires either a proof or refutation of the Overlap Rigidity Axiom.

\bigskip
\noindent
\textbf{Status:} Conditional, isolated, and formally anchored.

\end{document}

