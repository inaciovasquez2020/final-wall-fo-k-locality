\documentclass[11pt]{article}

\usepackage{amsmath,amssymb,amsthm}
\usepackage{geometry}
\geometry{margin=1in}

\title{Appendix A: Explicit FO$^4$ Formulas for Overlap Rigidity}
\author{Inacio F. Vasquez}
\date{}

\newtheorem{definition}{Definition}
\newtheorem{lemma}{Lemma}

\begin{document}
\maketitle

\section*{Purpose}

This appendix records the explicit first-order formulas used in the
\emph{Conditional Final Wall} theorem.
All formulas use at most four variables and are intended to be read
in the standard FO-with-adjacency signature for graphs.

\section{Basic Predicates}

We assume the signature:
\[
E(x,y) \quad \text{(undirected adjacency)}
\]

Distance predicates $\mathrm{dist}_{\le r}(x,y)$ are treated as FO-abbreviations,
unrolled as bounded existential chains.

\section{Non-Backtracking Walks}

\begin{definition}[Non-backtracking step]
A triple $(x,y,z)$ is non-backtracking if
\[
E(x,y) \wedge E(y,z) \wedge x \neq z.
\]
\end{definition}

\section{Simple Cycle Witness}

Fix a constant $L \ge 3$.

A simple cycle of length $\ell \le L$ rooted at $x$ is witnessed by
variables $(x_1,x_2,x_3,x_4)$ satisfying:

\[
\begin{aligned}
& E(x,x_1) \wedge E(x_1,x_2) \wedge E(x_2,x_3) \wedge E(x_3,x) \\
& \wedge\ x \neq x_2 \wedge x_1 \neq x_3 \wedge x_2 \neq x_4
\end{aligned}
\]

Longer cycles are encoded by chaining such guarded blocks.
FO$^4$ suffices because only local overlap, not full enumeration,
is required.

\section{Guarded Cycle Intersection}

\begin{definition}[Guarded intersection]
Two cycle witnesses intersect at radius $R_0$ around $v$ if
\[
\mathrm{dist}_{\le R_0}(v,x_i) \wedge \mathrm{dist}_{\le R_0}(v,y_j)
\]
for some vertices in each cycle witness.
\end{definition}

\section{Cycle Independence}

Cycle independence is encoded negatively:

Two cycles are dependent if one can be expressed as a parity sum
of the other inside the guarded neighborhood.
In FO$^4$, we approximate independence by forbidding
shared nontrivial subpaths of length $\ge 2$.

\section{Guarded Boundedness Formula}

\begin{definition}[Guarded boundedness $\psi_{R_0,m}$]
The formula $\psi_{R_0,m}$ asserts:

\begin{quote}
For every vertex $v$, there do not exist
$m+1$ pairwise independent simple cycle witnesses,
each intersecting $B_{R_0}(v)$.
\end{quote}

Formally:
\[
\forall v\; \neg \exists C_1,\dots,C_{m+1}
\Big(
\bigwedge_i \mathrm{Cycle}(C_i)
\wedge
\bigwedge_i \mathrm{Intersect}_{R_0}(v,C_i)
\wedge
\bigwedge_{i<j} \mathrm{Independent}(C_i,C_j)
\Big)
\]
where each $C_i$ is encoded by a 4-variable guarded block.
\end{definition}

\section{Overlap Detection Formula}

\begin{definition}[Overlap detection $\varphi_{R_0,m}$]
Let
\[
\varphi_{R_0,m} := \neg \psi_{R_0,m}.
\]

This formula asserts the existence of a vertex witnessing
local cycle overlap rank exceeding $m$.
\end{definition}

\section{EF-Game Interpretation}

The existence of $m+1$ overlapping independent cycles forces
a Spoiler win in the $k$-pebble EF-game within bounded rounds,
by cycling pebbles through incompatible neighborhoods.

\section*{Status}

All formulas in this appendix:
\begin{itemize}
\item use at most four variables,
\item are guarded by bounded-radius predicates,
\item align exactly with the Conditional Overlap Rigidity axiom,
\item are mirrored in the Lean formalization as abstract predicates.
\end{itemize}

This appendix completes the formal FO$^4$ specification of the Final Wall.

\end{document}

