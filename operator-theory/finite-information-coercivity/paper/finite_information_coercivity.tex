\documentclass[11pt]{article}
\usepackage{amsmath,amssymb,amsthm}

\newtheorem{theorem}{Theorem}
\newtheorem{lemma}{Lemma}

\title{Finite-Information Observation Implies Spectral Coercivity}
\author{Inacio F. Vasquez}
\date{\today}

\begin{document}
\maketitle

\begin{abstract}
We prove that for local self-adjoint operators, finite-information
observation channels imply spectral coercivity modulo finite defect.
The result is operator-theoretic and independent of computational
complexity assumptions.
\end{abstract}

\section{Introduction}
To be completed.

\section{Admissible Systems}

Let $\mathcal H$ be a separable Hilbert space and let $\mathsf L$ be a
densely defined, self-adjoint, nonnegative operator on $\mathcal H$.

We assume $\mathsf L$ is \emph{local} in the following sense: there exists
a decomposition of $\mathcal H$ into spatial degrees of freedom together
with constants $R<\infty$ and $C>0$ such that the resolvent
$(\mathsf L - z)^{-1}$ satisfies an exponential off-diagonal decay bound
uniformly for $\operatorname{Im} z \neq 0$.

Observation of the system is modeled by channels generated from bounded
functions of $\mathsf L$ with finite spacetime support.

\begin{definition}[Observation Capacity]
The information capacity of the system is
\[
\mathrm{Cap}(\mathsf L)
=
\sup_{\mathcal O} I(\rho;\mathcal O),
\]
where the supremum ranges over all admissible observation channels
$\mathcal O$ and normal states $\rho$.
\end{definition}

We say the system is \emph{finite-information observable} if
$\mathrm{Cap}(\mathsf L)<\infty$.
O

\section{Capacity--Coercivity Inequality}

Let
\[
N(\varepsilon;\mathsf L)
=
\dim \mathbf 1_{[0,\varepsilon)}(\mathsf L)\mathcal H
\]
denote the spectral counting function near zero.

\begin{theorem}[Capacity--Coercivity Inequality]
\label{thm:cci}
Let $\mathsf L$ be a local self-adjoint operator in dimension $d$ with
finite interaction range $R$. If $\mathrm{Cap}(\mathsf L)\le C<\infty$,
then there exist constants $\alpha=\alpha(d,R)$ and $K=K(d,R)$ such that
for all sufficiently small $\varepsilon>0$,
\[
N(\varepsilon;\mathsf L)
\le
K\, e^{C}\, \varepsilon^{-\alpha}.
\]
\end{theorem}

The constants $\alpha$ and $K$ depend only on the geometric locality
parameters and not on the particular state $\rho$ or channel $\mathcal O$.

\section{Spectral Coercivity Modulo Finite Defect}

\begin{theorem}[Unified Coercivity]
\label{thm:coercivity}
Under the assumptions of Theorem~\ref{thm:cci}, there exists a finite-rank
projection $P$ and a constant $c>0$ such that
\[
\langle \psi, \mathsf L \psi \rangle
\ge
c \|\psi\|^2
\quad
\text{for all }
\psi \perp \operatorname{Ran} P.
\]
\end{theorem}

In particular, the defect space
\[
\mathcal D(\mathsf L)
=
\ker \mathsf L
\oplus
\bigcap_{\varepsilon>0}
\operatorname{Ran}\mathbf 1_{[0,\varepsilon)}(\mathsf L)
\]
is finite dimensional.

\section{Admissible Systems}
To be completed.

\section{Capacity--Coercivity Inequality}
To be completed.

\end{document}

