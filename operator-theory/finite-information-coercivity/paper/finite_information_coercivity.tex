\documentclass[11pt]{article}

\usepackage{amsmath,amssymb,amsthm}
\usepackage{geometry}
\usepackage{hyperref}

\geometry{margin=1in}

\newtheorem{theorem}{Theorem}
\newtheorem{lemma}{Lemma}
\newtheorem{definition}{Definition}
\newtheorem{proposition}{Proposition}

\title{Finite-Information Observation Implies Spectral Coercivity}
\author{Inacio F. Vasquez}
\date{\today}

\begin{document}
\maketitle

\begin{abstract}
We prove an operator-theoretic result establishing that for local
self-adjoint operators, finite-information observation channels imply
spectral coercivity modulo finite defect. The theorem is independent of
computational complexity assumptions and applies to a broad class of
operators arising in spectral theory and mathematical physics.
\end{abstract}

\section{Introduction}

Locality plays a central role in modern spectral theory and mathematical
physics. In many settings, physical or operational constraints limit the
amount of information that can be extracted from a system through
admissible observations. This work formalizes the consequence of such
limitations: if the observation process is information-bounded and the
underlying operator is local, then the spectrum near zero must be rigid.

The result is purely operator-theoretic. No assumptions are made about
computational complexity classes, and no global complexity-theoretic
claims are asserted. Any algorithmic interpretations are explicitly
conditional and external to the present framework.

\section{Admissible Systems}

Let $\mathcal H$ be a separable Hilbert space and let
$\mathsf L:\mathcal D(\mathsf L)\subset\mathcal H\to\mathcal H$ be a
densely defined, self-adjoint, nonnegative operator.

\begin{definition}[Local Operator]
We say that $\mathsf L$ is local if there exists a decomposition of
$\mathcal H$ into spatial degrees of freedom together with constants
$R<\infty$ and $M>0$ such that the resolvent $(\mathsf L-z)^{-1}$ satisfies
an exponential off-diagonal decay bound for all $\operatorname{Im}z\neq0$.
\end{definition}

Observation of the system is modeled by channels generated from bounded
functions of $\mathsf L$ with finite spacetime support.

\begin{definition}[Observation Capacity]
The information capacity of $\mathsf L$ is defined by
\[
\mathrm{Cap}(\mathsf L)
=
\sup_{\rho,\mathcal O} I(\rho;\mathcal O),
\]
where the supremum ranges over all normal states $\rho$ and all admissible
observation channels $\mathcal O$.
\end{definition}

We say the system is \emph{finite-information observable} if
$\mathrm{Cap}(\mathsf L)<\infty$.

\section{Capacity--Coercivity Inequality}

Let
\[
N(\varepsilon;\mathsf L)
=
\dim \mathbf 1_{[0,\varepsilon)}(\mathsf L)\mathcal H
\]
denote the spectral counting function near zero.

\begin{theorem}[Capacity--Coercivity Inequality]
\label{thm:cci}
Let $\mathsf L$ be a local self-adjoint operator acting in dimension $d$
with finite interaction range $R$. If
$\mathrm{Cap}(\mathsf L)\le C<\infty$, then there exist constants
$\alpha=\alpha(d,R)$ and $K=K(d,R)$ such that for all sufficiently small
$\varepsilon>0$,
\[
N(\varepsilon;\mathsf L)
\le
K\,e^{C}\,\varepsilon^{-\alpha}.
\]
\end{theorem}

The constants $\alpha$ and $K$ depend only on geometric locality
parameters and are independent of the particular state or observation
channel.

\section{Spectral Coercivity Modulo Finite Defect}

\begin{theorem}[Unified Coercivity]
\label{thm:coercivity}
Under the assumptions of Theorem~\ref{thm:cci}, there exists a finite-rank
projection $P$ and a constant $c>0$ such that
\[
\langle \psi,\mathsf L\psi\rangle
\ge
c\|\psi\|^2
\quad
\text{for all }
\psi\perp\operatorname{Ran}P.
\]
\end{theorem}

Equivalently, the defect space
\[
\mathcal D(\mathsf L)
=
\ker \mathsf L
\oplus
\bigcap_{\varepsilon>0}
\operatorname{Ran}\mathbf 1_{[0,\varepsilon)}(\mathsf L)
\]
is finite dimensional.

\section{Discussion and Scope}

Theorems~\ref{thm:cci} and~\ref{thm:coercivity} establish a rigidity
principle linking locality and finite-information observation to spectral
structure. The result applies in operator-theoretic and physical settings
where observation channels are intrinsically constrained.

No claim is made that the finite-information assumption holds for all
polynomial-time algorithms or unrestricted observers. Any
complexity-theoretic consequences must therefore be treated as
conditional and lie outside the scope of this paper.

\appendix

\section{Explicit Constants and Locality Bounds}

We record explicit constants appearing in
Theorem~\ref{thm:cci}.

\subsection{Combes--Thomas Constants}

Assume $\mathsf L$ acts on $\ell^2(\mathbb Z^d)$ with interaction range
$R$ and bounded row sum
\[
\sup_x \sum_{|y-x|\le R}
|\langle\delta_x,\mathsf L\delta_y\rangle|
\le M.
\]

Then for $z=E+i\eta$ with $\eta\neq0$,
\[
|\langle\delta_x,(\mathsf L-z)^{-1}\delta_y\rangle|
\le
\frac{2}{|\eta|}
\exp\!\left(-\frac{|\eta|}{2MR}|x-y|\right).
\]

Thus one may take
\[
c=\frac{|\eta|}{2MR},
\qquad
C_0=\frac{2}{|\eta|}.
\]

\subsection{Covering Constants}

Let $B_r\subset\mathbb Z^d$ denote the $\ell^1$ ball of radius $r$. Then
\[
|B_r|\le (2r+1)^d.
\]

For a local operator of range $R$, the effective covering constants may be
chosen as
\[
C(d,R)=2^{dR^d},
\qquad
\alpha(d,R)=dR^d.
\]

These constants depend only on geometric locality parameters.

\end{document}

